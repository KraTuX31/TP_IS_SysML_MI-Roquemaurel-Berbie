\chapter*{À propos}

La \textbf{Maison Intelligente} (MI) constitue un lieu où technologies et sciences humaines se rencontrent pour trouver des solutions permettant d'aider à l'accompagnement du vieillissement des populations dans notre société (handicaps, dépendances, etc).
 
Pour ce faire, la Maison Intelligente propose un ensemble de solutions permettant à celle-ci de s'adapter à son habitant.  L'ensemble des possibilités qu'offre la MI sont définies dans le document : M2DL2015-ExigencesMI.
% (cf. https://docs.google.com/spreadsheets/d/1-yaW8fZHG9i6r7vJ7NOU9-66-hnDYY66Vd9cmZTYs2A/edit ). 


L'objectif du document est de proposer un ensemble de diagrammes définies au moyen de la norme SysML répondant aux exigences relatives à la \textbf{salle de bain} de la MI.

Les diagrammes ici représentés seront :
\begin{itemize}
	\item diagramme des \textbf{exigences} (req)
	\item diagramme des \textbf{cas d'utilisation} (uc)
	\item diagramme de \textbf{blocs de définition} (bdd)
	\item diagramme \textbf{interne de blocs} (idb)
	\item diagramme comportementaux
	\begin{itemize}
		\item diagramme d'\textbf{états} (st)
		\item diagramme de \textbf{séquences} (seq)
	\end{itemize}
\end{itemize} 